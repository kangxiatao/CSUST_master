% !Mode:: "TeX:UTF-8"

% 手动修改页眉
% \fancyhead[CO]{\song\wuhao ~致谢~}
% \fancypagestyle{plain}{\fancyhead[CO]{\song\wuhao ~致谢~}}

\addcontentsline{toc}{chapter}{致\quad 谢}%添加到目录中
\chapter*{致\quad 谢}

所谓的成长,不过是燃烧自己的无知,然后逐渐麻木。

恍惚间已是三年,伴随着疫情和动荡,在毕业之际,真心的感谢祖国,感谢父母,感谢抗疫前线的英雄们,感谢所有人的牺牲和付出换来形式的向好,让我们生活步入正轨,安稳的完成学业。

古之学者必有师,感谢我亦师亦友的导师李平教授。在路灯下讨论方案、在食堂外分析实验、在教室里推导公式、在民宿中撰写论文,这些画面一帧一帧的浮现出来,成了我研究生期间最重要的记忆。在李老师将下,我跟进了最前沿的技术、丰富了理论知识和思维、参加了学术会议,为我的人生写了一段关于科研魅力的故事。

风华正茂,感谢这些年挚友的关怀和帮助。特别感谢~OYX~为我提供丰富的高校资源,感谢~XZH~为我讲解算法和部署实践,感谢~KZC~在实验上提供的灵感和他仅有的显卡,感谢~AF~分享的文献和调参经验。

修辞立其诚,实验立其真。作为人工智能领域的学者,非常感谢~Kaggle~、~Google Colab~、阿里云天池、华为云AI提供的云计算资源。感谢~DeepMind~、~OpenAI~和~FAIR~提供的开源代码和技术文档。更感谢这几年所有前辈的努力,我们拥有一个很友好的深度学习开发环境。

为学生二十余载,感谢我终迎来了最后一次毕业。我很好学,但终究是做不来学者,这条路也算是走到头了。在研究生期间,在这个历史的冬天,在这个卷不赢躺不平的时代,我的哲思过于极端,以至于都懒得去建设短暂的同学情谊。很长一段时间我都在找一个答案,一个现代主义没有给出的答案。逐渐我也有了答案,本以为的麻木,殊不知为无用之用。

一树百获的三年,虽行久矣,而方始发,任重道远。或某日同风起,直上九万里。

\rightline{康夏涛}
\rightline{二零二三年四月}
\rightline{于长沙理工}









