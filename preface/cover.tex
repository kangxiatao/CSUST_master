% !Mode:: "TeX:UTF-8"

% """
% Edited on 03/10/2023
% anoi.
% @author: Kang Xiatao (kangxiatao@gmail.com)
% """

\chnunumer{10536}
\chnuname{长沙理工大学}
\cclassnumber{TP391}
\cnumber{20208051427}
\csecret{公开}
\cmajor{工程硕士}  % 学位类别
\cheading{专业硕士学位论文}      % 设置正文的页眉,以及自己的学位级别
\dtitle{神经网络剪枝与稀疏模型泛化研究}%封面用论文标题,自己可手动断行\\
\ctitle{神经网络剪枝与稀疏模型泛化研究}  %页眉标题无需断行
\etitle{Research on Neural Network Pruning and Sparse Model Generalization}
\caffil{计算机与通信工程学院} %学院名称
\csubjecttitle{学科专业}
\csubject{电子信息}   %学位领域
\cauthortitle{研究生}     % 学位
\cauthor{康夏涛}   %学生姓名
\ename{KANG~Xiatao}
\cbe{B.E.~(Changsha University of Science \& Technology)~2020}
% \cms{M.S.~(University)~2020}
\cdegree{thesis}
\cclass{Master of engineering}
% \emajor{Computer Science and Technology}
\ehnu{Changsha University of Science \& Technology}
\esupervisor{Li Ping}
\csupervisortitle{指导教师}
\csupervisor{李平~教授} %导师姓名
\elevel{Professor} %导师职称
\ocsupervisor{~} %校外导师
\cchair{~~~~~~~~}
\ddate{2023年6月} %论文答辩日期
\edate{April,~2023}

\untitle{长沙理工大学}
\declaretitle{学位论文原创性声明}
\declarecontent{
    本人郑重声明:所呈交的论文是本人在导师的指导下独立进行研究所取得的研究成果。除了文中特别加以标注引用的内容外,本论文不包含任何其他个人或集体已经发表或撰写的成果作品。对本文的研究做出重要贡献的个人和集体,均已在文中以明确方式标明。本人完全意识到本声明的法律后果由本人承担。
}
\authorizationtitle{学位论文版权使用授权书}
\authorizationcontent{
    本学位论文作者完全了解学校有关保留、使用学位论文的规定,同意学校保留并向国家有关部门或机构送交论文的复印件和电子版,允许论文被查阅和借阅。本人授权长沙理工大学可以将本学位论文的全部或部分内容编入有关数据库进行检索\scalebox{0.9}[1]{,}可以采用影印、缩印或扫描等复制手段保存和汇编本学位论文。同时授权中国科学技术信息研究所将本论文收录到《中国学位论文全文数据库》,并通过网络向社会公众提供信息服务。
}
\authorizationadd{本学位论文属于}
\authorsigncap{作者签名:}
\supervisorsigncap{导师签名:}
\signdatecap{日期:}


%\cdate{\CJKdigits{\the\year} 年\CJKnumber{\the\month} 月 \CJKnumber{\the\day} 日}
% 如需改成二零一二年四月二十五日的格式,可以直接输入,即如下所示
\cdate{2023年4月} %论文提交日期
% \cdate{\the\year 年\the\month 月 \the\day 日} % 此日期显示格式为阿拉伯数字 如2012年4月25日
\cabstract{
中文摘要应将学位论文的内容要点简短明了地表达出来,约500~800字左右(限一页),字体为宋体小四号。内容应包括工作目的、研究方法、成果和结论。要突出本论文的创新点,语言力求精炼。为了便于文献检索,应在本页下方另起一行注明论文的关键词(3-7个)。
}

\ckeywords{关键词1;~~关键词2;~~关键词3;~~关键词4}

\eabstract{
Externally pressurized gas bearing has been widely used in the field of aviation, semiconductor, weave, and measurement apparatus because of its advantage of high accuracy, little friction, low heat distortion, long life-span, and no pollution. In this thesis, based on the domestic and overseas researching……

}

\ekeywords{Key Word1;~~Key Word 2;~~Key Word 3;~~Key Word 4}

\makecover

\clearpage
